\documentclass[]{article}
\usepackage{lmodern}
\usepackage{amssymb,amsmath}
\usepackage{ifxetex,ifluatex}
\usepackage{fixltx2e} % provides \textsubscript
\ifnum 0\ifxetex 1\fi\ifluatex 1\fi=0 % if pdftex
  \usepackage[T1]{fontenc}
  \usepackage[utf8]{inputenc}
\else % if luatex or xelatex
  \ifxetex
    \usepackage{mathspec}
  \else
    \usepackage{fontspec}
  \fi
  \defaultfontfeatures{Ligatures=TeX,Scale=MatchLowercase}
\fi
% use upquote if available, for straight quotes in verbatim environments
\IfFileExists{upquote.sty}{\usepackage{upquote}}{}
% use microtype if available
\IfFileExists{microtype.sty}{%
\usepackage{microtype}
\UseMicrotypeSet[protrusion]{basicmath} % disable protrusion for tt fonts
}{}
\usepackage[margin=1in]{geometry}
\usepackage{hyperref}
\hypersetup{unicode=true,
            pdftitle={Portfolio 3},
            pdfauthor={Kiri Koppelgaard},
            pdfborder={0 0 0},
            breaklinks=true}
\urlstyle{same}  % don't use monospace font for urls
\usepackage{graphicx,grffile}
\makeatletter
\def\maxwidth{\ifdim\Gin@nat@width>\linewidth\linewidth\else\Gin@nat@width\fi}
\def\maxheight{\ifdim\Gin@nat@height>\textheight\textheight\else\Gin@nat@height\fi}
\makeatother
% Scale images if necessary, so that they will not overflow the page
% margins by default, and it is still possible to overwrite the defaults
% using explicit options in \includegraphics[width, height, ...]{}
\setkeys{Gin}{width=\maxwidth,height=\maxheight,keepaspectratio}
\IfFileExists{parskip.sty}{%
\usepackage{parskip}
}{% else
\setlength{\parindent}{0pt}
\setlength{\parskip}{6pt plus 2pt minus 1pt}
}
\setlength{\emergencystretch}{3em}  % prevent overfull lines
\providecommand{\tightlist}{%
  \setlength{\itemsep}{0pt}\setlength{\parskip}{0pt}}
\setcounter{secnumdepth}{0}
% Redefines (sub)paragraphs to behave more like sections
\ifx\paragraph\undefined\else
\let\oldparagraph\paragraph
\renewcommand{\paragraph}[1]{\oldparagraph{#1}\mbox{}}
\fi
\ifx\subparagraph\undefined\else
\let\oldsubparagraph\subparagraph
\renewcommand{\subparagraph}[1]{\oldsubparagraph{#1}\mbox{}}
\fi

%%% Use protect on footnotes to avoid problems with footnotes in titles
\let\rmarkdownfootnote\footnote%
\def\footnote{\protect\rmarkdownfootnote}

%%% Change title format to be more compact
\usepackage{titling}

% Create subtitle command for use in maketitle
\newcommand{\subtitle}[1]{
  \posttitle{
    \begin{center}\large#1\end{center}
    }
}

\setlength{\droptitle}{-2em}
  \title{Portfolio 3}
  \pretitle{\vspace{\droptitle}\centering\huge}
  \posttitle{\par}
  \author{Kiri Koppelgaard}
  \preauthor{\centering\large\emph}
  \postauthor{\par}
  \predate{\centering\large\emph}
  \postdate{\par}
  \date{September 28, 2018}


\begin{document}
\maketitle

\subsection{Welcome to the third exciting part of the Language
Development in ASD
exercise}\label{welcome-to-the-third-exciting-part-of-the-language-development-in-asd-exercise}

In this exercise we will delve more in depth with different practices of
model comparison and model selection, by first evaluating your models
from last time, then learning how to cross-validate models and finally
how to systematically compare models.

N.B. There are several datasets for this exercise, so pay attention to
which one you are using!

\begin{enumerate}
\def\labelenumi{\arabic{enumi}.}
\tightlist
\item
  The (training) dataset from last time (the awesome one you produced
  :-) ).
\item
  The (test) datasets on which you can test the models from last time:
\end{enumerate}

\begin{itemize}
\tightlist
\item
  Demographic and clinical data:
  \url{https://www.dropbox.com/s/ra99bdvm6fzay3g/demo_test.csv?dl=1}
\item
  Utterance Length data:
  \url{https://www.dropbox.com/s/uxtqqzl18nwxowq/LU_test.csv?dl=1}
\item
  Word data:
  \url{https://www.dropbox.com/s/1ces4hv8kh0stov/token_test.csv?dl=1}
\end{itemize}

\subsubsection{Exercise 1) Testing model
performance}\label{exercise-1-testing-model-performance}

How did your models from last time perform? In this exercise you have to
compare the results on the training data () and on the test data. Report
both of them. Compare them. Discuss why they are different.

\begin{itemize}
\item
  recreate the models you chose last time (just write the model code
  again and apply it to your training data (from the first assignment))
\item
  calculate performance of the model on the training data: root mean
  square error is a good measure. (Tip: google the function rmse())
\item
  create the test dataset (apply the code from assignment 1 part 1 to
  clean up the 3 test datasets)
\item
  test the performance of the models on the test data (Tips: google the
  functions ``predict()'')
\item
  optional: predictions are never certain, can you identify the
  uncertainty of the predictions? (e.g.~google predictinterval())
\end{itemize}

formatting tip: If you write code in this document and plan to hand it
in, remember to put include=FALSE in the code chunks before handing in.

REPONSE: The quadratic model (mean length of utterance \textasciitilde{}
visit + visit\^{}2 + diagnosis + (VISIT+ VISIT\^{}2\textbar{}SUBJ))
produces a root mean square error of 0.289, when explaining the train
data.The mean squared error increases when applying the model on the
test set and goes from 0.289 to 0.773. Since the error increases when
applied to the test set it appears the model does only has limited power
to predict and generalize to the population. This could be due to
overfitting of the data in the training set.

\subsubsection{Exercise 2) Model Selection via Cross-validation (N.B:
ChildMLU!)}\label{exercise-2-model-selection-via-cross-validation-n.b-childmlu}

One way to reduce bad surprises when testing a model on new data is to
train the model via cross-validation.

In this exercise you have to use cross-validation to calculate the
predictive error of your models and use this predictive error to select
the best possible model.

\begin{itemize}
\item
  Use cross-validation to compare your model from last week with the
  basic model (Child MLU as a function of Time and Diagnosis, and don't
  forget the random effects!)
\item
  (Tips): google the function ``createFolds''; loop through each fold,
  train both models on the other folds and test them on the fold)
\item
  Test both of them on the test data.
\item
  Report the results and comment on them.
\item
  Now try to find the best possible predictive model of ChildMLU, that
  is, the one that produces the best cross-validated results.
\item
  Which model is better at predicting new data: the one you selected
  last week or the one chosen via cross-validation this week?
\item
  Bonus Question 1: What is the effect of changing the number of folds?
  Can you plot RMSE as a function of number of folds?
\item
  Bonus Question 2: compare the cross-validated predictive error against
  the actual predictive error on the test data
\end{itemize}

RESPONSE: The quadratic model (mean length of utterance
\textasciitilde{} visit + visit\^{}2 + diagnosis + (VISIT+
VISIT\^{}2\textbar{}SUBJ)) produces a mean squared error of 0.773, when
applied to the test data compared to 0.658 produced by the best
explaining model from last time (mean length of utterance
\textasciitilde{} Diagnosis + visit+ visit\^{}2 + ADOS + verbal IQ +
nonverbal IQ + (VISIT+ I(VISIT\^{}2)\textbar{}SUBJ)).Thus, the best
model is still the last mentioned, which is able to predict the new data
the best. Thus, it does not appear that the fancy model does not overfit
as could have been expected.

Based on the mean of the mean squared errors, when applied to the
cross-validated test data, the best predictive model is mean length of
utterance \textasciitilde{} visit + diagnosis + MOT\_MLU +ADOS +
types\_CHI + tokens\_CHI + I(VISIT\^{}2)+(1+VISIT+
I(VISIT\^{}2)\textbar{}SUBJ). This also counts when the model is applied
to the `true' test data, there it produces a root mean square error of
0.47 compared to the best explaining model from last week, which has a
root mean square error of 0.66.

\subsubsection{Exercise 3) Assessing the single
child}\label{exercise-3-assessing-the-single-child}

Let's get to business. This new kiddo - Bernie - has entered your
clinic. This child has to be assessed according to his group's average
and his expected development.

Bernie is one of the six kids in the test dataset, so make sure to
extract that child alone for the following analysis.

You want to evaluate:

\begin{itemize}
\item
  how does the child fare in ChildMLU compared to the average TD child
  at each visit? Define the distance in terms of absolute difference
  between this Child and the average TD. (Tip: recreate the equation of
  the model: Y=Intercept+BetaX1+BetaX2, etc; input the average of the TD
  group for each parameter in the model as X1, X2, etc.).
\item
  how does the child fare compared to the model predictions at Visit 6?
  Is the child below or above expectations? (tip: use the predict()
  function on Bernie's data only and compare the prediction with the
  actual performance of the child)
\end{itemize}

Based on the best model from the cross-validation the typically
developed child starts of with a mean length of utterance on 2.17,
whereas Bernie starts 2.54. Thus, Bernie makes longer utterances when
entering the experiment. At the 6th visit Bernie has a mean length of
utterance of 3.17, whereas a typically developed child has a mean length
of utterance of 2.40. Thus, Bernie is pretty genius. He both starts of
with a higher MLU and develops faster, it would seem.

Based on the best predictive model found by cross-validation, Bernie is
estimated to having a mean length of utterance of 3.28 at visit 6, which
is relatively close to the true mean of 3.17. The model has slightly
overestimated him and predicts he has a faster development than he in
reality has, but it is close.

\subsubsection{OPTIONAL: Exercise 4) Model Selection via Information
Criteria}\label{optional-exercise-4-model-selection-via-information-criteria}

Another way to reduce the bad surprises when testing a model on new data
is to pay close attention to the relative information criteria between
the models you are comparing. Let's learn how to do that!

Re-create a selection of possible models explaining ChildMLU (the ones
you tested for exercise 2, but now trained on the full dataset and not
cross-validated).

Then try to find the best possible predictive model of ChildMLU, that
is, the one that produces the lowest information criterion.

\begin{itemize}
\tightlist
\item
  Bonus question for the optional exercise: are information criteria
  correlated with cross-validated RMSE? That is, if you take AIC for
  Model 1, Model 2 and Model 3, do they co-vary with their
  cross-validated RMSE?
\end{itemize}

\subsubsection{OPTIONAL: Exercise 5): Using Lasso for model
selection}\label{optional-exercise-5-using-lasso-for-model-selection}

Welcome to the last secret exercise. If you have already solved the
previous exercises, and still there's not enough for you, you can expand
your expertise by learning about penalizations. Check out this tutorial:
\url{http://machinelearningmastery.com/penalized-regression-in-r/} and
make sure to google what penalization is, with a focus on L1 and
L2-norms. Then try them on your data!


\end{document}
